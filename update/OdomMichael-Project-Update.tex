\documentclass[10pt,a4paper]{article}
\usepackage[decmulti]{inputenc}
\usepackage{amsmath}
\usepackage{amsfonts}
\usepackage{amssymb}
\usepackage{graphicx}
\graphicspath{ {./images/} }
\author{Michael Odom}
\title{HW00 CpE 590}
\begin{document}

\title{Project Update CpE 590}
\author{Michael Odom}
\date{\today}

\maketitle
\begin{center}
This project is currently in an exploratory phase of adding and subtracting different features from the dataset. The data set has a large quantity of features, some are not useful, as such I am attempting to find the minimum amount of features required to achieve these goals. \emph{Please note, after this point is an updated project proposal with additions emphasized.}
\end{center}
\begin{center}

In RF communications, the figure of merit for quality of communications is Signal-to-Noise-Ratio(SNR). Generally speaking, Power transmitted(Ptx) is the only parameter in SNR that is variable (the remainder of the parameters are either defined at construction or belong to the channel). A machine learning algorithm that is able to maintain minimum necessary SNR by varying Ptx. Minimizing this parameter provides a number of benefits, such as reducing power consumption or providing transmission security(Transec) to maintain privacy. The goal of this project is to engineer	an algorithm that can predict the minimum necessary transmit power, then classify it as either \emph{Acceptable SNR} or \emph{Unacceptable SNR} \emph{The plan is to use a nueral net with a sigmoid activation function to achieve this goal}.
\end{center}



\end{document}